%!TEX root = predecessor_search_example.tex

\begin{tikzpicture}[
    scale=1.5,
    % transform shape,
    every node/.style = {
      % draw
      % circle
    },
    grow = down,  % alignment of characters
    level 1/.style = {sibling distance=1cm},
    level 2/.style = {sibling distance=0.5cm,level distance=0.25cm},
    level 3/.style = {sibling distance=1cm},
    level distance = 0.5cm,
    minimum size=1pt
  ]

  \tikzset{
    cross/.style={
      cross out,
      draw,
      minimum size=2*(#1-\pgflinewidth),
      inner sep=0pt,
      outer sep=0pt
    },
    normalnode/.style = {
      circle,
      thick,
      scale = 2,
      minimum size=1pt,
      inner sep=0pt,
      outer sep=0pt,
      % mark size=1pt,
      % color=red,
      fill = black!90!black,
      % fill = orange!90!blue,
      % label = center:\textsf{\Large H}
    },
    active-node/.style = {
      normalnode,
      fill = red
    },
    empty-child/.style={
      sibling distance=0,
      level distance=0
    },
    active-edge/.style = {
      color = red,
      line width=1pt
    },
    inactive-edge/.style = {
      color = black,
      line width=0.5pt
    },
    empty-subtree/.style={
      scale=0.8,
      fill=gray,
      label={[label distance=-0.25cm]270:$\bot$}
    },
    subtree/.style={
      draw,
      color = black,
      fill = lightgray,
      fill opacity=0.8,
      % dashed,
      shape border uses incircle,
      isosceles triangle,
      shape border rotate=90,
      yshift=-0.5cm
    },
    query-arrow/.style={
      decorate,
      decoration={
        snake,
        amplitude=.05mm,
        segment length=0.5mm,
        post length=0.8mm
      }
    },
    query-node/.style={
      normalnode,
      scale = 1.5,
      darkgray
    },
    hline/.style={
      gray,
      thin,
      densely dashed
      %dashdotdotted
    }
  }

  \tikzset{
    itria/.style={
      draw,
      % dashed,
      shape border uses incircle,
      isosceles triangle,
      isosceles triangle apex angle=110,
      shape border rotate=90,
      % yshift=-1.45cm
      % yshift=-1.3cm
    },
    rtria/.style={
      draw,dashed,shape border uses incircle,
      isosceles triangle,isosceles triangle apex angle=90,
      shape border rotate=-45,yshift=0.2cm,xshift=0.5cm},
    ritria/.style={
      draw,dashed,shape border uses incircle,
      isosceles triangle,isosceles triangle apex angle=110,
      shape border rotate=-55,yshift=0.1cm},
    letria/.style={
      draw,dashed,shape border uses incircle,
      isosceles triangle,isosceles triangle apex angle=110,
      shape border rotate=235,yshift=0.1cm}
  }

  \node[normalnode, clockwise from=0, sibling angle=180, label={0:\scriptsize $u$}] (NodeU) {}
    child{
      node [normalnode, rectangle, label={0:\scriptsize $p_i+1$}] (NodePI) {}
      % [counterclockwise from=-120, sibling angle=60]
      child {
        node [normalnode, label={0:\scriptsize $v$}] (NodeV) {}
        node [subtree] (TrieV) {}
        edge from parent [inactive-edge] node [left] {}
      }
      child[empty-child] {
      }
      edge from parent [inactive-edge] node [left] {}
    }
    child[empty-child] {};
%,minimum size=1pt,inner sep=10pt,outer sep=0pt

  \begin{scope}[on background layer]
    \node [
      draw,
      color=black,
      anchor=north,
      shape=isosceles triangle,
      minimum height=1.025cm,
      minimum width=3cm,
      isosceles triangle stretches,
      shape border rotate=90,
      fill = lightgray!50!white
    ] (TriePI) at (NodePI) {};
  \end{scope}

  \node [anchor=south] at (TriePI.south east){\scriptsize $I_{p_i+1}$};
  \node [anchor=south] at (TrieV.south){\scriptsize $I_{v}$};
\end{tikzpicture}
