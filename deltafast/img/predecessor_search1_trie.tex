%!TEX root = predecessor_search1_example.tex

\begin{tikzpicture}[
    scale=1.5,
    % transform shape,
    every node/.style = {
      % draw
      % circle
    },
    grow = down,  % alignment of characters
    level 1/.style = {sibling distance=1cm},
    level 2/.style = {sibling distance=1cm},
    level 3/.style = {sibling distance=1cm},
    level distance = 0.75cm,
    minimum size=1pt
  ]

  \tikzset{
    cross/.style={
      cross out,
      draw,
      minimum size=2*(#1-\pgflinewidth),
      inner sep=0pt,
      outer sep=0pt
    },
    normalnode/.style = {
      circle,
      thick,
      scale = 2,
      minimum size=1pt,
      inner sep=0pt,
      outer sep=0pt,
      % mark size=1pt,
      % color=red,
      fill = black!90!black,
      % fill = orange!90!blue,
      % label = center:\textsf{\Large H}
    },
    active-node/.style = {
      normalnode,
      fill = red
    },
    empty-child/.style={
      sibling distance=0,
      level distance=0
    },
    active-edge/.style = {
      color = red,
      line width=1pt
    },
    inactive-edge/.style = {
      color = black,
      line width=0.5pt
    },
    empty-subtree/.style={
      scale=0.8,
      fill=gray,
      label={[label distance=-0.25cm]270:$\bot$}
    },
    subtree/.style={
      draw,
      color = black,
      fill = lightgray,
      fill opacity=0.8,
      % dashed,
      shape border uses incircle,
      isosceles triangle,
      shape border rotate=90,
      yshift=-0.5cm
    },
    query-arrow/.style={
      decorate,
      decoration={
        snake,
        amplitude=.05mm,
        segment length=0.5mm,
        post length=0.8mm
      }
    },
    query-node/.style={
      normalnode,
      scale = 1.5,
      darkgray
    },
    hline/.style={
      gray,
      thin,
      densely dashed
      %dashdotdotted
    }
  }

  \tikzset{
    itria/.style={
      draw,
      % dashed,
      shape border uses incircle,
      isosceles triangle,
      isosceles triangle apex angle=110,
      shape border rotate=90,
      % yshift=-1.45cm
      % yshift=-1.3cm
    },
    rtria/.style={
      draw,dashed,shape border uses incircle,
      isosceles triangle,isosceles triangle apex angle=90,
      shape border rotate=-45,yshift=0.2cm,xshift=0.5cm},
    ritria/.style={
      draw,dashed,shape border uses incircle,
      isosceles triangle,isosceles triangle apex angle=110,
      shape border rotate=-55,yshift=0.1cm},
    letria/.style={
      draw,dashed,shape border uses incircle,
      isosceles triangle,isosceles triangle apex angle=110,
      shape border rotate=235,yshift=0.1cm},
    show curve controls/.style={
        decoration={
            show path construction,
            curveto code={
                \draw [blue, dashed]
                    (\tikzinputsegmentfirst) -- (\tikzinputsegmentsupporta)
                    node [at end, cross out, draw, solid, red, inner sep=2pt]{};
                \draw [green, dashed]
                    (\tikzinputsegmentsupportb) -- (\tikzinputsegmentlast)
                    node [at start, cross out, draw, solid, red, inner sep=2pt]{};
            }
        }, decorate
    }
  }

% http://tex.stackexchange.com/questions/18617/tikz-draw-only-a-certain-central-length-of-a-given-path
\pgfdeclaredecoration{ignore}{final}
{
\state{final}{}
}

% Declare the actual decoration.
\pgfdeclaremetadecoration{prefix}{initial}{
    \state{initial}[
        width={1pt},
        next state=middle
    ]
    {\decoration{moveto}}
    \state{middle}[
        width={\the\pgfdecorationsegmentlength-1pt},
        next state=final
    ]
    {\decoration{curveto}}
    \state{final}
    {\decoration{ignore}}
}


\tikzset{
  position/.style={
    decoration={
      markings,
      mark=at position 0.6 with {\coordinate (NodePK);},
      mark=at position 0.8 with {\coordinate (NodeLCP);}
    },
    postaction={decorate}
  },
  trie/.style={
      draw,
      color = black,
      shape=isosceles triangle,
      minimum height=3.5cm,
      minimum width=5cm,
      isosceles triangle stretches,
      shape border rotate=90,
      inner sep=0,
      fill = lightgray!50!white,
      fill opacity=0.8,
  },
  initial segment/.style={
    decoration={prefix},
    decorate,
    segment length=#1
  }
}

    \node [trie] (triangle) at (0,0) {};

    \coordinate (A) at ($ (triangle.north) !.25! (triangle.south) + (0.125, 0)$) ;
    \coordinate (B) at ($ (triangle.north) !.75! (triangle.south) - (0.125, 0)$) ;
    \coordinate (C) at ($ (triangle.left corner) !.7! (triangle.right corner)$) ;

    \coordinate (NodeQ) at ($ (triangle.left corner) !.35! (triangle.right corner)$) ;

    \coordinate (trianglestop0) at ($(triangle.north)-(0,0.5pt)$);
    \coordinate (trianglestop5) at ($(triangle.right corner)+(0,0.125pt)$);
    \coordinate (trianglestop1) at ($(trianglestop0) !.125! (trianglestop5)$);
    \coordinate (trianglestop2) at ($(trianglestop0) !.25! (trianglestop5)$);
    \coordinate (trianglestop3) at ($(trianglestop0) !.5! (trianglestop5)$);
    \coordinate (trianglestop4) at ($(trianglestop0) !.75! (trianglestop5)$);

    \coordinate (triangleleftstop0) at (trianglestop0);
    \coordinate (triangleleftstop5) at ($(triangle.left corner)-(0,1pt)$);
    \coordinate (triangleleftstop1) at ($(trianglestop0) !.125! (triangleleftstop5)$);
    \coordinate (triangleleftstop2) at ($(trianglestop0) !.25! (triangleleftstop5)$);
    \coordinate (triangleleftstop3) at ($(trianglestop0) !.5! (triangleleftstop5)$);
    \coordinate (triangleleftstop4) at ($(trianglestop0) !.75! (triangleleftstop5)$);

    \path[fill=gray!50!white]
          (triangle.north) -- (triangleleftstop3) -- (trianglestop3) -- cycle;

    \draw [red, position] plot [draw,smooth] coordinates {(trianglestop0) (A) (B) (C)};
    \draw [red,double,initial segment=1.825cm] plot [draw,smooth] coordinates {(trianglestop0) (A) (B) (C)};

    \path[draw] (triangleleftstop1) -- (trianglestop1);
    \path[draw] (triangleleftstop2) -- (trianglestop2);
    \path[draw] (triangleleftstop3) -- (trianglestop3);
    \path[draw] (triangleleftstop4) -- (trianglestop4);

    \coordinate (A2) at ($ (NodeLCP) !.25! (NodeQ) - (1pt, 0)$) ;
    \coordinate (B2) at ($ (NodeLCP) !.75! (NodeQ) + (1pt, 0)$) ;

    \draw [red] plot [draw,smooth] coordinates {(NodeLCP) (A2) (B2) (NodeQ)};

    \node[normalnode, label=0:\scriptsize $lcp(q)$] at (NodeLCP) {};
    \node[normalnode, label=0:\scriptsize $p_k$] at (NodePK) {};
    \node[normalnode, label=270:\scriptsize $q$] at (NodeQ) {};
    \node[normalnode] at (C) {};
    \node[normalnode] at (trianglestop0) {};


    \path (triangleleftstop1) -- (trianglestop1)
          node[normalnode,pos=0.725,label={[label distance=-0.5em]225:\scriptsize $a_1$}] {};

    \path (triangleleftstop3) -- (trianglestop3)
          node[normalnode,pos=0.48,label={[label distance=-0.5em]55:\scriptsize $a_{j}$}] {}
          node[normalnode,pos=0.48,label={[label distance=-0.5em]315:\scriptsize $lcp'$}] {};

    \path (triangleleftstop4) -- (trianglestop4)
          node[normalnode,pos=0.45,label={[label distance=-0.5em]225:\scriptsize $a_{j+1}$}] {};

    \node [trie, fill = none] (triangle) at (0,0) {};

    \coordinate (startscale) at ($(triangle.right corner) + (0.25,0)$);
    \coordinate (endscale) at ($(startscale) + (0, 15)$);

    \coordinate (scalestop0) at ($(startscale)!(trianglestop0)!(endscale)$);
    \coordinate (scalestop1) at ($(startscale)!(trianglestop1)!(endscale)$);
    \coordinate (scalestop2) at ($(startscale)!(trianglestop2)!(endscale)$);
    \coordinate (scalestop3) at ($(startscale)!(trianglestop3)!(endscale)$);
    \coordinate (scalestop4) at ($(startscale)!(trianglestop4)!(endscale)$);
    \coordinate (scalestop5) at ($(startscale)!(trianglestop5)!(endscale)$);

    \path[draw,<->] (scalestop0) -- (scalestop1) node[midway, right] {\scriptsize$w/2$};
    \path[draw,<->] (scalestop1) -- (scalestop2) node[midway, right,xshift=0.5em,yshift=0.3em] {\scriptsize$\vdots$};
    \path[draw,<->] (scalestop2) -- (scalestop3) node[midway, right] {\scriptsize$w/2^{j}$};
    \path[draw,<->] (scalestop3) -- (scalestop4) node[midway, right] {\scriptsize$w/2^{j+1}$};
    \path[draw,<->] (scalestop4) -- (scalestop5) node[midway, right] {\scriptsize$w/2^{j+1}$};

    \path[draw,dotted] (trianglestop0) -- (scalestop0);
    \path[draw,dotted] (trianglestop1) -- (scalestop1);
    \path[draw,dotted] (trianglestop2) -- (scalestop2);
    \path[draw,dotted] (trianglestop3) -- (scalestop3);
    \path[draw,dotted] (trianglestop4) -- (scalestop4);
    \path[draw,dotted] (trianglestop5) -- (scalestop5);

\end{tikzpicture}
